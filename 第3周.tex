\documentclass{article}
\usepackage{CJKutf8}
\usepackage{minted}
\usepackage{geometry}
\geometry{a4paper,centering,scale=0.8}
\usepackage{graphicx}
\usepackage{amsmath}
\usepackage{textcomp}
\usepackage{amsthm}
\usepackage{amssymb}
\usepackage{float}
%可能用到的包
\title{Machine Learning - Week 3}
\author{赵燕}
\date{}
\begin{document} 
\hfuzz=\maxdimen
\tolerance=10000
\hbadness=10000
\begin{CJK}{UTF8}{gbsn} 
\maketitle
\renewcommand\contentsname{目录}
\renewcommand\figurename{图}
\tableofcontents
\newpage

\section{Classification and Representation}
\subsection{Classification}
\subparagraph{}
在分类问题中,需要预测的变量y是离散的值,引出要学习的逻辑回归算法(Logistic Regression),这是目前最流行使用的一种学习算法。
\subparagraph{}
分类问题举例:
\subparagraph{}
(1)判断一封电子邮件是否是垃圾邮件;
\subparagraph{}
(2)判断一次金融交易是否是欺诈;
\subparagraph{}
(3)判断肿瘤是 良性还是恶性;
\begin{figure}[H]
\center{\includegraphics[width=.8\textwidth]{301.png}}
\caption{分类问题举例}
\label{fig:301}
\end{figure}
\subparagraph{}
从二元的问题开始讨论:
\subparagraph{}
将因变量(dependent variable)可能属于两个类分别称为负向类(negative class)和正向类(positive class),则因变量$y\in{\{0,1}\}$,其中0表示负向类,1表示正向类。
\begin{figure}[H]
\center{\includegraphics[width=.8\textwidth]{302.png}}
\caption{图示}
\label{fig:302}
\end{figure}
\subparagraph{}
如果我们要用线性回归算法来解决一个分类问题,对于分类,y 取值为 0 或者 1,但如果你使用的是线性回归,那么假设函数的输出值可能远大于 1,或者远小于 0,即使所有训练样本的标签 y 都等于 0 或 1。尽管我们知道标签应该取值 0 或者 1,但是如果算法得到的值远大于 1 或者远小于 0 的话,就会感觉很奇怪。所以我们在接下来的要研究的算法就叫做逻辑回归算法,这个算法的性质是:它的输出值永远在 0 到 1 之间。
\begin{figure}[H]
\center{\includegraphics[width=.8\textwidth]{303.png}}
\caption{逻辑回归算法}
\label{fig:303}
\end{figure}
\subparagraph{}
逻辑回归算法是分类算法,我们将它作为分类算法使用,有时候可能因为这个算法的名字中出现了“回归”使你感到困惑,但逻辑回归算法实际上是一种分类算法,它适用于标签y取值离散的情况下,如:1 ,0, 0, 1 。
\subsection{Hpothesis Representation}
\subsection{Decision Boundary}

\section{Logistic Regression Model}
\subsection{Cost Function}
\subsection{Simplified Cost Function and Gradient Descent}
\subsection{Advanced Optimization} 

\section{Multiclass Classification}
\subsection{Multiclass Classification:One-vs-all}
\subparagraph*{}
\end{CJK}
\end{document}