\documentclass{article}
\usepackage{CJKutf8}
\usepackage{minted}
\usepackage{geometry}
\geometry{a4paper,centering,scale=0.8}
\usepackage{graphicx}
\usepackage{amsmath}
\usepackage{textcomp}
\usepackage{amsthm}
\usepackage{amssymb}
\usepackage{float}
%可能用到的包
\title{Machine Learning - Week 3}
\author{赵燕}
\date{}
\begin{document} 
\hfuzz=\maxdimen
\tolerance=10000
\hbadness=10000
\begin{CJK}{UTF8}{gbsn} 
\maketitle
\renewcommand\contentsname{目录}
\renewcommand\figurename{图}
\tableofcontents
\newpage

\section{Motivations}
\subsection{Non-linear Hypotheses}
\subparagraph{}
我们之前学过的线性回归和逻辑回归都有一个缺点:当特征量太多时,计算的负荷会非常大。
\subparagraph{}
例如:
\begin{figure}[H]
\center{\includegraphics[width=.8\textwidth]{402.png}}
\caption{举例说明}
\label{fig:402}
\end{figure}
\subparagraph{}
当我们使用$x_1$和$x_2$的多项式进行预测时,我们可以应用的很好。
\subparagraph{}
之前我们已经看到过,使用非线性的多项式项,能够帮助我们建立更好的分类模型。假设我们有非常多的特征,例如大于 100 个变量,我们希望用这 100 个特征来构建一个非线性
的多项式模型,结果将是数量非常惊人的特征组合,即便我们只采用两两特征的组合($x_1x_2+x_1x_3+x_1x_4+...+x_2x_3+x_2x_4+...+x_99x_100$) ,我们也会有接近 5000个组合而成的特征。这对于一般的逻辑回归来说需要计算的特征太多了。
\subparagraph{}
假设我们希望训练一个模型来识别视觉对象(例如识别一张图片上是否是一辆汽车),我么怎样才能这么做呢?一种方法是我们利用很多汽车的图片和很多非汽车的图片,然后利用这些图片上的一个个像素的值(饱和度或亮度)来作为特征。
\subparagraph{}
假如我们只使用灰度图片,每个像素则只有一个值(而非
RGB值),我们可以选取图片上的两个不同位置的两个像素,然后训练一个逻辑回归算法,利用这两个像素的值来判断图片上是否是汽车:
\begin{figure}[H]
\center{\includegraphics[width=.8\textwidth]{403.png}}
\caption{计算机视觉-汽车模型}
\label{fig:403}
\end{figure}
\subparagraph{}
假使我们采用的都是50x50像素的小图片,并且我们将所有的像素视为特征,则会有2500个特征,如果我们要进一步将两两特征组合构成一个多项式模型,则会有约$\frac{2500^2}{2}$ 个(接近3百万个)特征。普通的逻辑回归模型,不能有效地处理这么多的特征,这时候我们需要神经网络。
\subsection{Neurons and the Brain}
\subparagraph{}
神经网络是一种很古老的算法,它最初产生的目的是制造能模拟大脑的机器。它是计算量有些偏大的算法,然而大概由于近些年计算机的运行速度变快,才足以运行起大规模的神经网络。当想模拟大脑时,是指想制造出与人类大脑作用效果相同的机器。大脑可以学会去以看而不是听的方式处理图像,学会处理我们的触觉。
\section{Neural Networks}
\subsection{Model Representation I}
\subparagraph{}
\subsection{Model Representation II}
\section{Applications}
\subsection{Examples and Intuitions I}
\subsection{Examples and Intuitions II}
\subsection{Multiclass Classification}

\end{CJK}
\end{document}