\documentclass{article}
\usepackage{CJKutf8}
\usepackage{minted}
\usepackage{geometry}
\geometry{a4paper,centering,scale=0.8}
\usepackage{graphicx}
\usepackage{amsmath}
\usepackage{textcomp}
\usepackage{amsthm}
\usepackage{amssymb}
\usepackage{float}
%可能用到的包
\title{Machine Learning - Week 3}
\author{赵燕}
\date{}
\begin{document} 
\hfuzz=\maxdimen
\tolerance=10000
\hbadness=10000
\begin{CJK}{UTF8}{gbsn} 
\maketitle
\renewcommand\contentsname{目录}
\renewcommand\figurename{图}
\tableofcontents
\newpage

\section{Motivations}
\subsection{Non-linear Hypotheses}
\subparagraph{}
\subsection{Neurons and the Brain}
\subparagraph{}
\section{Neural Networks}
\subsection{Model Representation I}
\subsection{Model Representation II}
\section{Applications}
\subsection{Examples and Intuitions I}
\subsection{Examples and Intuitions II}
\subsection{Multiclass Classification}
在接下来的课程中,我们将会学习一个非常强大的非线性分类器,无论是线性回归问题,还是逻辑回归问题,都可以构造多项式来解决。你将会逐渐发现还有更强大的非线性分类器,可以用来解决多项式回归问题,我们接下来将学会比现在解决问题的方法强大N倍的学习算法。
\end{CJK}
\end{document}