\documentclass{article}
\usepackage{CJKutf8}
\usepackage{minted}
\usepackage{geometry}
\geometry{a4paper,centering,scale=0.8}
\usepackage{graphicx}
\usepackage{amsmath}
\usepackage{textcomp}
\usepackage{amsthm}
\usepackage{amssymb}
\usepackage{float}
%可能用到的包
\title{Machine Learning - Week 8}
\author{赵燕}
\date{}
\begin{document} 
\hfuzz=\maxdimen
\tolerance=10000
\hbadness=10000
\begin{CJK}{UTF8}{gbsn} 
\maketitle
\renewcommand\contentsname{目录}
\renewcommand\figurename{图}
\tableofcontents
\newpage

\part{Unsupervised Learning}
\section{Clustering}
\subparagraph{}
聚类算法:非监督学习算法,要让计算机学习无标签数据,而不是之前的标签数据。
\subsection{Unsupervised Learning:Introduction}
\subparagraph{}
监督学习和无监督学习的比较:
\subparagraph{}
在一个监督学习中,我们有一个有标签的训练集,我们的目标是找到能够区分正样本和负样本的决策边界,我们的监督学习中,有一系列标签,我们需要据此拟合一个假设函数。
\begin{figure}[H]
\center{\includegraphics[width=.3\textwidth]{802.png}}
\label{fig:802}
\caption{监督学习}
\end{figure}
\subparagraph{}
与此不同的是,在无监督学习中,数据没有附带任何标签,我们拿到的数据就是这样的:
\begin{figure}[H]
\center{\includegraphics[width=.3\textwidth]{801.png}}
\label{fig:801}
\caption{无监督学习}
\end{figure}
\subparagraph{}
在这里我们有一系列的点,却没有任何标签,因此我们的训练集可以写成只有$x_1,x_2,x_3,...,x_m$,没有任何标签y。因此图上画的这些点没有标签信息,也就是说,在无监督学习中,我们需要将一系列无标签的训练数据,输入到下一个算法中,然后我们告诉这个算法,快去为我们找找这个数据的内在结构给定数据。可能需要某种算法帮助我们寻找一种结构。图上的数据看起来可以分成两个分开的点集(称为簇),一个能够找到我圈出的这些点集的算法,被称为聚类算法。
\subparagraph{}
聚类算法使我们介绍的第一个无监督学习算法,以后还将提到其他的无监督学习算法,它可以为我们找到其他类型的结构或者其他的一些模式,而不是簇。
\subparagraph{}
聚类算法是用来做什么的?
\begin{figure}[H]
\center{\includegraphics[width=.4\textwidth]{803.png}}
\label{fig:803}
\caption{聚类算法的应用}
\end{figure}
\begin{itemize}
\item 市场分割:也许你在数据库中分别销售产品或者分别提供更合适的服务,而你希望他们分为不同的课客户群;
\item 社交网络分析:例如 Facebook,Google+,或者是其他的一些信息,比如说:你经常跟哪些人联系,而这些人又经常给哪些人发邮件,由此找到关系密切的人群,因此,这可能需要另一个聚类算法,你希望用它发现社交网络中关系密切的朋友。
\item 管理数据中心:使用聚类算法来更好的组织计算机集群,或者更好的管理数据中心,因为数据中心中,了解了哪些计算机数据中心更倾向于一起协作工作,那么就可以重新分配资源,重新布局网络,由此优化数据中心,优化数据通信。
\item 天文学:利用聚类算法视图了解星系的形成和其中的天文学的具体细节。
\end{itemize}
\begin{figure}[H]
\center{\includegraphics[width=.7\textwidth]{805.png}}
\label{fig:805}
\end{figure}
\subsection{K-Means Algorithm}
\subparagraph{}
K-均值算法是最普及的聚类算法,算法接受一个未标记的数据集,然后将数据聚类成不同的组。
\subparagraph{}
执行K-均值算法,首先先随机选择两个点,这两个点叫做聚类中心,就是图上的两个叉,为什么是两个点呢,因为我们希望聚出两个类。
\subparagraph{}
K-均值是一个迭代方法,需要做两件事情:第一是簇分配,第二是移动聚类中心。
\subparagraph{}
簇分配:需要遍历所有的样本,就是图上绿色的点,然后依据每一个点是更接近红色的这个中心,还是更接近绿色的这个中心,来将每个数据点分配到两个不同的聚类中心中。
\begin{figure}[H]
\center{\includegraphics[width=.3\textwidth]{806.png}}
\label{fig:806}
\caption{簇分配}
\end{figure}
\subparagraph{}
具体来讲,就是对对数据集中的所有点,根绝他们更接近红色这个中心,还是蓝色这个中心,进行染色,染色的结果如图:
\begin{figure}[H]
\center{\includegraphics[width=.3\textwidth]{807.png}}
\label{fig:807}
\caption{簇分配染色}
\end{figure}
\subparagraph{}
移动聚类中心:将两个聚类中心,也就是说红色的叉和蓝色的叉移动到 和它一样颜色的那堆点的均值处,找出所有红色的点,计算出它们的均值,就是所有红色的点平均下来的位置,然后把红色点的聚类中心移动到这里,蓝色也是这样,找出所有蓝色的点计算它们的均值把蓝色的叉放到那里,我们将按照图上所示这么移动:
\begin{figure}[H]
\center{\includegraphics[width=.3\textwidth]{808.png}}
\label{fig:808}
\caption{移动聚类中心}
\end{figure}
\subparagraph{}
重复上述步骤,继续簇分配和移动聚类中心,一直迭代下去,直到最后聚类中心不会变,并且哪些点的颜色也不会变,在这时我们就可以说 K-均值方法已经收敛了,在这些数据中找到两个簇。
\subparagraph{}
K-均值是一个迭代算法,假设我们想要将数据聚类成n个组,其方法是:
\subparagraph{}
首先选择K个随机的点,称为聚类中心(cluster centroids);
\subparagraph{}
对于数据集中的每一个数据,按照距离K个中心点的距离,将其与距离最近的中心点关联起来,与同一个中心点关联的所有点聚成一类。
\subparagraph{}
计算每一个组的平均值,将该组所关联的中心点移动到平均值的位置。
\subparagraph{}
重复步骤直至中心点不再变化。
\begin{figure}[H]
\center{\includegraphics[width=.4\textwidth]{809.png}}
\label{fig:809}
\caption{K-均值算法}
\end{figure}
\subparagraph{}
$c^{(i)}$ 是距离样本$x^{(i)}$ 的距离的平方,使距离最小或者距离的平方最小都能让我们得到一个相同的$c^{(i)}$,注意$c^{(i)}$ 是一个在1到K之间的数。(通常还是写成距离的平方)
\begin{figure}[H]
\center{\includegraphics[width=.7\textwidth]{810.png}}
\label{fig:810}
\caption{K-均值算法}
\end{figure}
\subparagraph{}
算法分为两个步骤,第一个 for 循环是赋值步骤,即:对于每一个样例 i,计算其应该属于的类。第二个 for 循环是聚类中心的移动,即:对于每一个类 k,重新计算该类的聚类中心。
\subparagraph{}
K-均值算法也可以很便利地用于将数据分为许多不同组,即使在没有非常明显区分的组群的情况下也可以。下图所示的数据集包含身高和体重两项特征构成的,利用 K-均值算法将数据分为三类,用于帮助确定将要生产的T-恤衫的三种尺寸 S,M,L。
\begin{figure}[H]
\center{\includegraphics[width=.7\textwidth]{812.png}}
\label{fig:812}
\caption{K-means for non-separated clusters}
\end{figure}
\begin{figure}[H]
\center{\includegraphics[width=.8\textwidth]{811.png}}
\label{fig:811}
\end{figure}
\subsection{Optimization Objective}
\subsection{Random Initialization}
\subsection{Choosing the Number of Clusters}
\part{Dimensionality Reduction}
\section{Motivation}
\subsection{Motivation I:Compression}
\subsection{Motivation II:Visualization}
\section{Principal Component Analysis}
\subsection{Principal Component Analysis Problem Formulation}
\subsection{Principal Component Analysis Algorithm}
\section{Applying PCA}
\subsection{Reconstruction from Compressed Representation}
\subsection{Choosing the Number of Principal Components}
\subsection{Advice for Applying PCAfg}
\end{CJK}
\end{document}