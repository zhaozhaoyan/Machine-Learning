\documentclass{article}
\usepackage{CJKutf8}
\usepackage{minted}
\usepackage{geometry}
\geometry{a4paper,centering,scale=0.8}
\usepackage{graphicx}
\usepackage{amsmath}
\usepackage{textcomp}
\usepackage{amsthm}
\usepackage{amssymb}
\usepackage{float}
%可能用到的包
\title{Machine Learning - Week 6}
\author{赵燕}
\date{}
\begin{document} 
\hfuzz=\maxdimen
\tolerance=10000
\hbadness=10000
\begin{CJK}{UTF8}{gbsn} 
\maketitle
\renewcommand\contentsname{目录}
\renewcommand\figurename{图}
\tableofcontents
\newpage

\section{Evalutaing a Learning Algorithm}
\subsection{Deciding What to Try Next}
\begin{figure}[H]
\center{\includegraphics[width=.8\textwidth]{601.png}}
\label{fig:601}
\end{figure}
\subparagraph{}
机器学习诊断法:
\begin{figure}[H]
\center{\includegraphics[width=.8\textwidth]{602.png}}
\caption{机器学习诊断法}
\label{fig:602}
\end{figure}
\begin{figure}[H]
\center{\includegraphics[width=.8\textwidth]{603.png}}
\label{fig:603}
\end{figure}
\subsection{Evaluating a Hypothesis}
\subparagraph{}
怎样用你学过的算法来评估假设函数。
\begin{figure}[H]
\center{\includegraphics[width=.8\textwidth]{604.png}}
\label{fig:604}
\end{figure}
\subparagraph{}
当确定学习算法的参数的时候,我们考虑的是选择参量来使训练误差最小化,有人
认为得到一个非常小的训练误差一定是一件好事,但已经知道,仅仅是因为这个假设具
有很小的训练误差,并不能说明它就一定是一个好的假设函数。而且也学习了过拟合假
设函数的例子,所以这推广到新的训练集上是不适用的。
\subparagraph{}
那么,该如何判断一个假设函数是过拟合的呢?对于这个简单的例子,我们可以对假
设函数 h(x) 进行画图,然后观察图形趋势,但对于特征变量不止一个的这种一般情况,还
有像有很多特征变量的问题,想要通过画出假设函数来进行观察,就会变得很难甚至是不可
能实现。
\subparagraph{}
因此,需要另一种方法来评估我们的假设函数过拟合检验。
\begin{figure}[H]
\center{\includegraphics[width=.8\textwidth]{604.png}}
\caption{训练集和测试集}
\label{fig:604}
\end{figure}
\subparagraph{}
\begin{figure}[H]
\center{\includegraphics[width=.6\textwidth]{606.png}}
\label{fig:606}
\end{figure}
\subparagraph{}
按如下步骤训练和测试学习算法:
\begin{figure}[H]
\center{\includegraphics[width=.8\textwidth]{608.png}}
\caption{训练集和测试集}
\label{fig:608}
\end{figure}
\subsubsection{Evaluating a Hypothesis}
\subparagraph{}
Once we have done some trouble shooting for errors in our predictions by: 
\begin{itemize}
\item Getting more training examples
\item Trying smaller sets of features
\item Trying additional features
\item Trying polynomial features
\item Increasing or decreasing λ
\end{itemize}
\subparagraph{}
We can move on to evaluate our new hypothesis. 
\subparagraph{}
A hypothesis may have a low error for the training examples but still be inaccurate (because of overfitting). Thus, to evaluate a hypothesis, given a dataset of training examples, we can split up the data into two sets: a training set and a test set. Typically, the training set consists of 70 \%{} of your data and the test set is the remaining 30 \%{}. 
\paragraph{}
The new procedure using these two sets is then:
\subparagraph{}
1.Learn $\Theta$ and minimize $J_{train}(\Theta)$ using the training set
\subparagraph{}
2.Compute the test set error $J_{test}(\Theta)$
\subsubsection{The test set error}
\subparagraph{}
1.For linear regression,利用测试集数据计算代价函数:
\begin{equation}
J_{test}(\Theta)=\frac{1}{2m_{test}}\sum_{i=1}^{m_{test}}(h_\Theta(x_{test}^{(i)})-y_{test}^{(i)})^2
\end{equation}
\subparagraph{}
2.For classification ~ Misclassification error (aka 0/1 misclassification error):(误分率)
\begin{figure}[H]
\center{\includegraphics[width=.6\textwidth]{609.png}}
\label{fig:609}
\end{figure}
\subparagraph{}
然后对计算结果求平均。
\subparagraph{}
This gives us a binary 0 or 1 error result based on a misclassification. The average test error for the test set is:
\begin{equation}
{Test}\quad{Error}=\frac{1}{m_{test}}\sum_{i=1}^{m_{test}}err(h_\Theta(x_{test}^{(i)}),y_{test}^{(i)})
\end{equation}
\subparagraph{}
This gives us the proportion of the test data that was misclassified.
\subsection{Model Section and Train/Validation/Test Sets}
\section{Bias vs. Variance}
\subsection{Diagnosing Bias vs. Variance}
\subsection{Regularization and Bias/Variance}
\subsection{Learning Curves}
\subsection{Deciding What to Do Next Revisited}
\section{Building a Spam Classifier}
\subsection{Prioritizing What to Work On}
\subsection{Error Analysis}
\section{Handing Skewed Data}
\subsection{Error Metrics for Skewed Classes}
\subsection{Trading Off Precision and Recall}
\section{Using Large Data Sets}
\subsection{Data For Machine Learning}
\end{CJK}
\end{document}