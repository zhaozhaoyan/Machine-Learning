\documentclass{article}
\usepackage{CJKutf8}
\usepackage{minted}
\usepackage{geometry}
\geometry{a4paper,centering,scale=0.8}
\usepackage{graphicx}
\usepackage{amsmath}
\usepackage{textcomp}
\usepackage{amsthm}
\usepackage{amssymb}
\usepackage{float}
%可能用到的包
\title{Machine Learning - Week 3}
\author{赵燕}
\date{}
\begin{document} 
\hfuzz=\maxdimen
\tolerance=10000
\hbadness=10000
\begin{CJK}{UTF8}{gbsn} 
\maketitle
\renewcommand\contentsname{目录}
\renewcommand\figurename{图}
\tableofcontents
\newpage

\section{Classification and Representation}
\subsection{Classification}
\subparagraph{}
To attempt classification, one method is to use linear regression and map all predictions greater than 0.5 as a 1 and all less than 0.5 as a 0. However, this method doesn't work well because classification is not actually a linear function.
\subparagraph{}
在分类问题中,需要预测的变量y是离散的值,引出要学习的逻辑回归算法(Logistic Regression),这是目前最流行使用的一种学习算法。
\subparagraph{}
分类问题举例:
\subparagraph{}
(1)判断一封电子邮件是否是垃圾邮件;
\subparagraph{}
(2)判断一次金融交易是否是欺诈;
\subparagraph{}
(3)判断肿瘤是 良性还是恶性;
\begin{figure}[H]
\center{\includegraphics[width=.8\textwidth]{301.png}}
\caption{分类问题举例}
\label{fig:301}
\end{figure}
\subparagraph{}
从二元的问题开始讨论:
\subparagraph{}
将因变量(dependent variable)可能属于两个类分别称为负向类(negative class)和正向类(positive class),则因变量$y\in{\{0,1}\}$,其中0表示负向类,1表示正向类。
\begin{figure}[H]
\center{\includegraphics[width=.8\textwidth]{302.png}}
\caption{图示}
\label{fig:302}
\end{figure}
\subparagraph{}
The classification problem is just like the regression problem, except that the values we now want to predict take on only a small number of discrete values. For now, we will focus on the binary classification problem in which y can take on only two values, 0 and 1. (Most of what we say here will also generalize to the multiple-class case.) For instance, if we are trying to build a spam classifier for email, then x(i) may be some features of a piece of email, and y may be 1 if it is a piece of spam mail, and 0 otherwise. Hence, y∈{0,1}. 0 is also called the negative class, and 1 the positive class, and they are sometimes also denoted by the symbols “-” and “+.” Given x(i), the corresponding y(i) is also called the label for the training example.
\subparagraph{}
如果我们要用线性回归算法来解决一个分类问题,对于分类,y 取值为 0 或者 1,但如果你使用的是线性回归,那么假设函数的输出值可能远大于 1,或者远小于 0,即使所有训练样本的标签 y 都等于 0 或 1。尽管我们知道标签应该取值 0 或者 1,但是如果算法得到的值远大于 1 或者远小于 0 的话,就会感觉很奇怪。所以我们在接下来的要研究的算法就叫做逻辑回归算法,这个算法的性质是:它的输出值永远在 0 到 1 之间。
\begin{figure}[H]
\center{\includegraphics[width=.8\textwidth]{303.png}}
\caption{逻辑回归算法}
\label{fig:303}
\end{figure}
\subparagraph{}
逻辑回归算法是分类算法,我们将它作为分类算法使用,有时候可能因为这个算法的名字中出现了“回归”使你感到困惑,但逻辑回归算法实际上是一种分类算法,它适用于标签y取值离散的情况下,如:1 ,0, 0, 1 。
\subsection{Hpothesis Representation}
\subparagraph{}
假设函数表达式:
\subparagraph{}
我们希望分类器的输出在0和1之间,因此,要想出一个满足某个性质的假设函数,这个性质是它的预测值要在0和1之间。
\subparagraph{}
回顾肿瘤分类问题:可以用线性回归的方法求出一条适合数据的一条直线:
\begin{figure}[H]
\center{\includegraphics[width=.8\textwidth]{304.png}}
\caption{肿瘤分类线性回归}
\label{fig:304}
\end{figure}
\subparagraph{}
根据线性回归我们只能预测连续的值,然而对于分类问题,我们要输出0或1,我们可以预测:
(1)当$h_\theta$大于等于 0.5 时,预测 y=1;
(2)当 $h_\theta$ 小于 0.5 时,预测 y=0 对于上图所示的数据,这样的一个线性模型似乎能很好地完成分类任务。假使我们又观测到一个非常大尺寸的恶性肿瘤,将其作为实例加入到我们的训练集中来,这将使得我们获得一条新的直线。
\begin{figure}[H]
\center{\includegraphics[width=.8\textwidth]{305.png}}
\caption{肿瘤问题线性回归超范围}
\label{fig:305}
\end{figure}
\subparagraph{}
这时,再使用 0.5 作为阀值来预测肿瘤是良性还是恶性便不合适了。可以看出线性回归模型,因为其预测的值可以超越[0,1]的范围,并不适合解决这样的问题。
\subparagraph{}
We could approach the classification problem ignoring the fact that y is discrete-valued, and use our old linear regression algorithm to try to predict y given x. However, it is easy to construct examples where this method performs very poorly. Intuitively, it also doesn’t make sense for hθ(x) to take values larger than 1 or smaller than 0 when we know that y ∈ {0, 1}. To fix this, let’s change the form for our hypotheses hθ(x) to satisfy 0≤hθ(x)≤1. This is accomplished by plugging θTx into the Logistic Function.
\subparagraph{}
引入一个新的模型,逻辑回归,该模型的输出变量范围始终在 0 和 1 之间。
\subparagraph{}
逻辑回归模型的假设是:
\begin{equation}
h_\theta(x)=g(\theta^Tx)
\end{equation}
其中:
\subparagraph{}
x:代表特征向量;
\subparagraph{}
g:代表逻辑函数(logistic function),是一个常用的逻辑函数为 S 形函数(Sigmoid function),公式为:
\begin{equation}
g(z)=\frac{1}{1+e^{-z}}
\end{equation}
\subparagraph{}
该函数的图像为:
\begin{figure}[H]
\center{\includegraphics[width=.8\textwidth]{306.png}}
\caption{逻辑函数(S函数)图像}
\label{fig:306}
\end{figure}
\begin{figure}[H]
\center{\includegraphics[width=.8\textwidth]{308.png}}
\caption{逻辑函数表达式及图像}
\label{fig:308}
\end{figure}
\subparagraph{}
The function g(z), shown here, maps any real number to the (0, 1) interval, making it useful for transforming an arbitrary-valued function into a function better suited for classification.
\subparagraph{}
逻辑回归模型的假设:
\begin{equation}
h_\theta(x)=\frac{1}{1+e^{-\theta^Tx}}
\end{equation}
\subparagraph{}
$h_\theta(x)$的作用是,对于给定的输入变量,根据选择的参数计算输出变量=1 的可能性(estimated probablity)即
\begin{equation}
h_\theta(x)=P(y=1|x;\theta)
\end{equation}
\subparagraph{}
$h_\theta(x)$will give us the probability that our output is 1. 
\subparagraph{}
\begin{equation}
h_\theta(x)=P(y=1|x;\theta)=1-P(y=0|x;\theta)
\end{equation}
\begin{equation}
P(y=0|x;\theta)+P(y=1|x;\theta)=1
\end{equation}
例如,如果对于给定的 x,通过已经确定的参数计算得出$h_\theta(x)=0.7$,则表示有 70\%{}的几率 y 为正向类,相应地 y 为负向类的几率为 1-0.7=0.3。
\subsection{Decision Boundary}
\subparagraph{}
决策边界(Decision Boundary)可以更好的帮助我们理解假设函数在计算什么
\subparagraph{}
首先回顾一下逻辑回归中假设函数的表达式:
\begin{figure}[H]
\center{\includegraphics[width=.8\textwidth]{309.png}}
\caption{逻辑函数回顾}
\label{fig:309}
\end{figure}
\subparagraph{}
根据上面绘制的S形函数的图像,我们可以知道:
\subparagraph{}
当z=0时,g(z)=0.5;
\subparagraph{}
当z$>$0时,g(z)$>$0.5;
\subparagraph{}
当z$<$0时,g(z)$<$0.5;
\subparagraph{}
又因为
\begin{equation}
z=\theta^Tx
\end{equation}
\subparagraph{}
所以:
\subparagraph{}
$\theta^Tx\geq0$时,预测y=1;
\subparagraph{}
$\theta^Tx<0$时,预测y=0;
\subparagraph{}
假设有一个模型:
\begin{figure}[H]
\center{\includegraphics[width=.8\textwidth]{310.png}}
\caption{线性决策边界}
\label{fig:310}
\end{figure}
\subparagraph{}
绘制直线$x_1+x_2=3$,这条直线便是我们的模型的分界线,将预测y=1的区域和预测y=0的区分隔开,如图所示:
\begin{figure}[H]
\center{\includegraphics[width=.4\textwidth]{320.png}}
\caption{线性决策边界图}
\label{fig:320}
\end{figure}
\subparagraph{}
另一种情况:
\begin{figure}[H]
\center{\includegraphics[width=.8\textwidth]{312.png}}
\caption{非线性决策边界}
\label{fig:312}
\end{figure}
\subparagraph{}
需要二次方特征,得到的决策边界(判定边界)恰好是在原点且半径为1的圆形。
\subparagraph{}
我们可以用非常复杂的模型来适应非常复杂的判定边界。
例如:
\begin{equation}
h_\theta(x)=g(\theta_0+\theta_1x_1+\theta_2x_2+\theta_3x_1^2+\theta_4x_1^2x_2+\theta_5x_1^2x_2^2+\theta_6x_1^3x_2+...)
\end{equation}
\subparagraph{}
我们可能会得到比椭圆等更复杂的判定边界:
\begin{figure}[H]
\center{\includegraphics[width=.4\textwidth]{313.png}}
\caption{更复杂的情况}
\label{fig:313}
\end{figure}
\section{Logistic Regression Model}
\subsection{Cost Function}

\subsection{Simplified Cost Function and Gradient Descent}
\subsection{Advanced Optimization} 

\section{Multiclass Classification}
\subsection{Multiclass Classification:One-vs-all}
\subparagraph*{}
\end{CJK}
\end{document}